\documentclass[hoptionsi]{acmart}
\usepackage[utf8]{inputenc}
\usepackage{float}
\title{Nombre del proyecto}
\author{director }
\date{Fecha}

\begin{document}

\maketitle

\section{Introducción}
\section{Propósito y justificación del proyecto}
\section{Alcance}
\section{Marco teórico}
\section{Descripción del proyecto}
\subsection{Objetivos}
\begin{itemize}
    \item Alcance
    \item Cronograma 
    \item Indicador de éxito
\end{itemize}
\subsection{Entrégables}
\subsection{Premisas y restricciones}
\subsection{Riesgos iniciales de alto nivel}
\subsection{Cronogramas de hitos principales}

\begin{table}[h]
\begin{tabular}{|l|l|}
\hline
\textbf{Hito} & \textbf{Fecha tope} \\ \hline
              &                     \\ \hline
\end{tabular}
\end{table}

\section{Resumen del proyecto}

\begin{table}[H]
\begin{tabular}{|l|l|l|l|l|}
\hline
Objetivo & Actividades & \begin{tabular}[c]{@{}l@{}}Fechas de\\ inicio y fin\end{tabular} & Responsable & \begin{tabular}[c]{@{}l@{}}Indicadores\\ de éxito\end{tabular} \\ \hline
& &  &  &  \\ \hline

\end{tabular}
\end{table}


\section{Recursos}
\subsection{Humanos}
\subsection{Infraestructura}
\subsection{Materiales y equipos}
\subsection{Otros}
\section{Requisitos de apreciación del proyecto}
\section{Criterios de cierre o cancelación}



\end{document}
